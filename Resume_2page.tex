\documentclass[a4paper,10pt]{article}
%-----------------------------------------------------------
% {{{
\usepackage[top=0.5in, bottom=0.3in, left=0.5in, right=0.5in]{geometry}
\usepackage{graphicx}
\usepackage{hyperref}
\usepackage{url}
\usepackage{palatino}
\fontfamily{SansSerif}
\selectfont

\usepackage[T1]{fontenc}
\usepackage[ansinew]{inputenc}
% \usepackage{helvetica}
% \usepackage{array}
\usepackage{color}
\definecolor{mygrey}{gray}{0.75}
\textheight=10.4in
\raggedbottom
% \raggedright
\setlength{\tabcolsep}{0in}
\newcommand{\isep}{-2 pt}
\newcommand{\lsep}{-0.5cm}
\newcommand{\psep}{-0.5cm}
\renewcommand{\labelitemii}{$\circ$}
% Adjust margins
%\addtolength{\oddsidemargin}{-0.375in}
%\addtolength{\evensidemargin}{-0.375in}
%\addtolength{\textwidth}{1.75in}
%\addtolength{\topmargin}{0.1375in}
%\addtolength{\textheight}{1.75in}
\pagestyle{empty}
% }}}
%-----------------------------------------------------------
%Custom commands
\newcommand{\resitem}[1]{\item #1 \vspace{-2pt}}
\newcommand{\resheading}[1]{{\small
        {
            \begin{minipage}
                {0.992\textwidth}\textbf{{\textsc{#1 \vphantom{p\^{E}} }}}
                \\[-0.3cm]
                \hrule
            \end{minipage}
            \\[-0.5cm]
        }
 }}
\newcommand{\ressubheading}[3]{
\begin{tabular*}{6.62in}{l @{\extracolsep{\fill}} r}
    \textsc{{\textbf{#1}}} & \textsc{\textit{[#2]}} \\
\end{tabular*}\vspace{-8pt}}
% \textit{#3} &  \\
%-----------------------------------------------------------

\begin{document}
%{{{
\hspace{0.2cm}\\
\hspace{0.2cm}\\
\hspace{0.2cm}\\
\hspace{0.2cm}\\
\hspace{0.2cm}\\
\hspace{0.2cm}\\
\hspace{0.2cm}\\
\hspace{0.2cm}\\
\hspace{0.2cm}\\
\hspace{0.2cm}\\
\hspace{0.2cm}\\
% \hspace{0.2cm}\\
% \hspace{0.2cm}\\
% \\
\hspace{0.5cm}\\
% \hspace{0.2cm}\\
% \hspace{0.2cm}\\
%}}}
\resheading{\textbf{\large Technical Skills}}\\[\lsep]
\\ [-0.2cm]
%{{{
    \begin{itemize}\itemsep \isep
        \item \textbf{Programming \& Scripting Languages:}  C++, Python, MATLAB and R
        \item \textbf{Tools and Technologies:} NumPy, Pandas, PyTorch, TensorFlow, Keras, Scikit-learn, OpenCV
        \item \textbf{Machine Learning \& Other Skills:} Classification, Neural Networks, CNN, Data Structures \& Algorithms
    \end{itemize}
%}}}
\resheading{\textbf{\large Mtech Research}}\\[\lsep] 
% {{{
\\ [-0.2cm]
\begin{itemize}
\item \textbf{Defeaturing of CAD models using Deep Learning}  \\
    \emph{M.Tech Project, Guide: Prof. Prof. S.S.Pande} \hfill \emph{(May'19-till date)}
    \\[-0.6cm]
    \begin{itemize}\itemsep \isep
        \item \textbf{Objective:} To extract features from CAD model and study its effect on Finite Element Analysis simulation
	    \item Developing system to generate 10K 3D models with distinct topological features in Python
	    \item Extracting features from CAD models using concept based \textbf{3D Convolution Neural Network}
	    \item Aiming to reduce simulation computational time using \textbf{autoencoder} and \textbf{principal component analysis}
    \end{itemize}

\item \textbf{Application of Machine Learning in CAD/CAM} \\
    \emph{M.Tech Seminar, Guide:  Prof. S.S.Pande}
    \hfill \emph{(Jan'19-Apr'19)}
    \\ [-0.6cm]
      \begin{itemize}\itemsep \isep
        \item Carried out literature survey of machine learning application in CAD/CAM and explored its implementation %methodologies
     \item Studied research papers proposing applications such as feature recognition, defeaturing of CAD models
    \item Explored CNN for analysis of 3D models and inspected key drawbacks present in the implementation
	\end{itemize}
\end{itemize}
% }}}
\resheading{\textbf{\large Machine Learning Projects}}\\[\lsep]
\\[-0.2cm]
%{{{
\begin{itemize}

\item \textbf{Machine Learning based Image Classification System to Analyze Changing Fashion Trends} \\(\emph{Course Project, Foundations of Machine Learning, Advisor: Prof. Ganesh Ramakrishnan)\hfill (Jul'18-Nov'18)}
    \\ [-0.6cm]
    \begin{itemize}\itemsep \isep
    \item \textbf{Objective:} To classify shirts into various classes such as T-shirt, V-neck, Collar T-shirt classes
    \item Pre-processing of videos involved face detection and neck region feature to create dataset by using \textbf{OpenCV}
    \item Developed the shirt classification system using \textbf{KNN, SVM, CNN} with \textbf{Scikit-learn} and \textbf{TensorFlow} libraries
    \item Achieved the \textbf{accuracy of 84\%} for all the classes using \textbf{AlexNet} architecture as a base framework
    \end{itemize}

\item \textbf{Development of ML Algorithm for Flood Prediction on Azure Cloud Service} \\
\emph{(Microsoft Codefundo++)} \hfill \emph{(July'18-Oct'18)}
    \\ [-0.6cm]
    \begin{itemize}\itemsep \isep
  \item \textbf{Objective:} Designing and deploying ML workflow (flood prediction) on \textbf{Azure Cloud Services}
 %methodologies
    \item Dataset gleaned from Indian meteorology websites comprised of features like historical rainfall \& altitude
     \item Successfully completed all three stages and implemented web application on Azure Cloud Services
    \end{itemize}

\item \textbf{GPU accelerated implementation of Machine Learning algorithms using CUDA} \\
(\emph{Course Project, High Performance Scientific Computing, Advisor: Prof. Shiva Gopalakrishnan)\hfill (Jan'19-Apr'19)}
    \\ [-0.6cm]
    \begin{itemize}\itemsep \isep
    \item \textbf{Objective:} To implement the parallelization of \textbf{k-fold cross validation} for regression and classification
    \item Defined CUDA kernels for \textbf{Linear Regression} and \textbf{Logistic Regression} using Gradient Descent algorithm
    \item Achieved speed up of  \textbf{3.5X} for regression and \textbf{2.5X} for classification as compared with serial code
    \end{itemize}

\item \textbf{Implementation of Neural Network based Classifier} \\ 
(\emph{Course Project, Foundations of Machine Learning, Advisor: Prof. Ganesh Ramakrishnan)\hfill (July'18-Nov'18)}
    \\ [-0.6cm]
    \begin{itemize}\itemsep \isep
   \item \textbf{Objective:} Implement Neural Network from scratch to classify the Facebook comments into 5 categories %using the Back Propagation Algorithm
    \item Implemented NN architecture using \textbf{NumPy, Pandas} library and trained using Back Propagation Algorithm
    \item Improved accuracy by using different activation functions along with hyper-parameters tuning
\end{itemize}

\item \textbf{Pattern Recognition among Data using Machine Learning}\\ 
(\emph{Course Project, Engineering Data Mining and Applications, Advisor: Prof. Vinay Kulkarni)\hfill (July'18-Nov'18)}
\\[-0.6cm]
    \begin{itemize}\itemsep \isep
% 	\item Data preprocessing involved handling missing data, feature selection, normalization \& one hot encoding  %using the Back Propagation Algorithm
    % \item Compared accuracy of strategies like \textbf{Bagging, Boosting} \& \textbf{Random Forests} on test dataset
    \item Data preprocessing involved replacing missing data using regression tree
    and KNN 
    \item  Compared there performance using R-square, Adj. R, P-value, F-Stat and root mean square error
    \end{itemize}
\end{itemize}
\resheading{\textbf{\large Technical Projects}}\\[\lsep] 
% {{{
\\ [-0.2cm]
\begin{itemize}
\item \textbf{Mahindra Rise Driverless Car Challenge}  \\
    (\emph{Innovation Cell, IIT Bombay)\hfill (Dec'18-Apr'19)}
    \\[-0.6cm]
    \begin{itemize}\itemsep \isep
        \item Part of a team of \textbf{20 members} aiming to build Self Driving Car; India's \textbf{1st} driverless car
    \item One of the \textbf{11 finalists} out of 259 teams (IV Level) and received a \textbf{Mahindra E2O Car} for further development
     \item Headed the mechatronics subsystem to design mechanisms to mount LIDAR, Camera on the car 
    \item Led fabrication of movable mechanism to mount \textbf{3D LIDAR} on the car to scan the environment 
    \end{itemize}

\item \textbf{Voxelization of 3D model and Cutting Forces Prediction} \\
\emph{(Course Project, Computer Graphics and Product Modelling, Advisor: Prof. S.S.Pande)\hfill (Jul'18-Nov'18)}
    \\ [-0.6cm]
      \begin{itemize}\itemsep \isep
  \item \textbf{Objective:} To develop voxelization algorithm of 3D CAD model for visualization using \textbf{OpenGL}
    \item Predicted the cutting forces and material removal rate during machining using voxelized CAD model
	\end{itemize}
% \end{itemize}
% }}}    
    
% \pagebreak

% \vspace{-0.2cm}
% \pagebreak
% \item \textbf{Devanagari Character Recognition using Feed Forward Neural Network in Tensorflow} \\ {\emph{(CS621: Artificial Intelligence, Guide: Prof. G. Sivakumar)}} \hfill {\emph{(November 2017)}}
%     \\ [-0.6cm]
%     \begin{itemize}\itemsep \isep
%     \item Objective was to recognize 46 handwritten Devanagari characters present in 28 x 28 PNG image using NN
%     \item Achieved \textbf{98.68\%} accuracy after training a single hidden layer neural network on 92000 samples
%     \end{itemize}

\end{itemize}
%}}}
%  \vspace{0.05cm}
\resheading{\textbf{\large Work Experience}}\\[\lsep]
\\[-0.2cm]
% {{{
\begin{itemize}
    \item \textbf{CEAT Ltd, Vadodara}\\
\emph{(Graduate Engineer Trainee)} \hfill \emph{(Jul'16-Jul'17)}
        \\ [-0.6cm]
        \begin{itemize}\itemsep \isep
	\item Worked with design and product development team to provide analysis led design solution
	\item Increased \textbf{productivity by 5\%} through implementing projects such as \textbf{optimization} of tire and its components
	\item Studied the effect of friction on contact patch area and dimensions of tire
% 	\item Implemented projects such as optimization of turn up height, Effect of Friction on Foot Print shape and dimensions which results in increasing Productivity by 5\%
	\item Hands on experience to do the simulation and post-processing of tire with scripting in \textbf{Abaqus} software
	%\item \emph{Current work:} Studied various existing dependency parsers and implementing a neural network dependency parser to validate the verb structure by feeding the different set of requirements from different programs
   % \item \emph{Future work:} Disambiguate the high-level and low-level natural language requirements. Perform an automatic requirement analysis (consistency and completeness) using NuXMV, PVS, SLDV, and Astree
    \end{itemize}
\end{itemize}
% }}}
% \\[-.6cm]
\vspace{-0.1cm}
\resheading{\textbf{\large Courses Taken }}\\[\lsep]
\\[-0.2cm]
%{{{
\begin{itemize}
    \item[] \begin{tabular}{p{9cm}p{9cm}l }
    Foundations of Machine Learning & Robotics\\ 
    Engineering Data Mining and Applications(Audit) & Computer Graphics \& Product Modelling \\
    % \end{tabular}
    % \vspace{-0.25cm}
    % \item[] \begin{tabular}{p{11cm}p{11cm}l }
    High Performance Scientific Computing & Mathematical Methods in Engineering
    \end{tabular}
\end{itemize}
%}}} 
% \pagebreak
\resheading{\textbf{\large Positions of Responsibility}}\\[\lsep] 
\\[-0.2cm]
%{{{
\begin{itemize}\itemsep \isep
  \item \textbf{Teaching Assistant, IIT Bombay} \emph{(Prof. S.S.Pande)}\\[-0.6cm]
    	\begin{itemize}
            \item \textbf{Computer Graphics and Product Modelling} \hfill\emph{(Apr'19-till date)}
            % \begin{itemize}
                % \item Mentored students, invigilation and evaluation of Graded Labs, Exams
                \\
                 Assisted Bachelor and Master students to clear their difficulties, also helping the professor in evaluation
                % ng assignments and examinations.
            % \end{itemize}
        	\item \textbf{Materials Processing and Simulation Laboratory} \hfill\emph{(Jul'18-till date)} 
        	\\
        Worked in a team to mentor students, evaluated exams and provided assistance in the course
    	\end{itemize}
  \item \textbf{Mentor ITSP, IIT Bombay} \hfill\emph{(May'19-July'19)}\\[-0.6cm]
    \begin{itemize}\itemsep \isep
         \item Guided \textbf{8 students} on the topics OCR recognition, handwritten character recognition using \textbf{Deep Learning}
            \item Provided the basic training of Python and Machine Learning to students
            \end{itemize}
     \item \textbf{Campus Ambassador, InterviewBit} \hfill\emph{(May'19-till date)}\\[-0.6cm]
    \begin{itemize}
          \item Organized coding competitions to help students for campus placement preparation
            \end{itemize}
            
\end{itemize}
% \vspace{-0.1cm}
%}}}
\resheading{\textbf{\large Achievements \& Extracurricular Activities}}\\[\lsep]
\\[-0.2cm]
\begin{itemize}\itemsep \isep
     \item Secured \textbf{Gold level} position in the \textbf{2019 WorldQuant Challenge} organised by WorldQuant VRC \hfill\emph{(2019)}
    
    \item Achieved \textbf{Rank 1 on Kaggle} among 112 students for Machine Learning Challenge \hfill\emph{(2018)}
    \item Secured \textbf{Department Rank 1} among 210 students of UG 2016 batch \hfill\emph{(2016)}
    \item Secured \textbf{Department Rank 3} in Diploma of 2012 batch \hfill\emph{(2016)}
    \item Represented IIT Bombay in \textbf{Microsoft Codefundo++} and completed all three stages of the competition \hfill\emph{(2018)}
   \item Attended 3 days \textbf{GPU} bootcamp using CUDA conducted by \textbf{NVIDIA} \hfill\emph{(2019)}
   
    \item Scored \textbf{99.31} percentile in GATE 2018 ME among \textbf{194,496} candidates\hfill\emph{(2018)}

    \item  Completed \textbf{Neural Networks and Deep Learning} course by DeepLearning.ai on Coursera \hfill\emph{(2019)}
    \item Volunteered for \textbf{Python Workshop} conducted by PG Academic Council \hfill\emph{(2018)}
    \item Participated in \textbf{AVISHKAR} Zonal level Research Project Competition organized by Pune University \hfill\emph{(2015)}
\end{itemize}
\resheading{\textbf{\large Interest \& Hobbies}}\\[\lsep] 
\\[-0.2cm]
%{{{
\begin{itemize}
 \item Swimming, Badminton, Cricket
\end{itemize}

\end{document}

